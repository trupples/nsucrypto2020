\documentclass[11pt]{llncs}
\usepackage[a4paper]{geometry}
\usepackage{mathtools}
\usepackage{amsfonts}
\usepackage{amsmath}
\usepackage{hyperref}
\usepackage[dvipsnames]{xcolor}
\hypersetup{
    colorlinks=true,
    linkcolor=RoyalPurple,   
    urlcolor=RoyalPurple
}

\title{NSUCRYPTO 2020: POLY}

\author{
	Ioan Dragomir\inst{1} \and
	Gabriel Tulba-Lecu\inst{2} \and
	Mircea-Costin Preoteasa\inst{3}
}

\institute{
	\email{ioandr@gomir.pw} \textendash \ Technical University of Cluj-Napoca \and
	\email{gabi\_tulba\_lecu@yahoo.com} \textendash \ Polytechnic Univeristy of Bucharest \and
	\email{mircea\_costin84@yahoo.com} \textendash \ Polytechnic Univeristy of Bucharest
}

\begin{document}
\let\oldaddcontentsline\addcontentsline
\def\addcontentsline#1#2#3{}
\maketitle
\def\addcontentsline#1#2#3{\oldaddcontentsline{#1}{#2}{#3}}


\let\oldnewpage\newpage
\def\newpage{\hfill}
\setcounter{tocdepth}{2}
\tableofcontents
\def\newpage{\oldnewpage}

\section{Problem summary}

Let $f$ be a polynomial of degree $n$ in $\mathbb{Z}$: $f(x) = a_0 + a_1x + a_2x^2 + \dots + a_nx^n$\\
Bob claims $f(20) = 7$ and $f(15) = 5$.\\
Prove that this is impossible.

\section{Solution}

\begin{align*}
      f(20) &= a_0 + a_1 \cdot 20 + a_2 \cdot 20^2 + \dots + a_n \cdot 20^n = 7 \\
      f(15) &= a_0 + a_1 \cdot 15 + a_2 \cdot 15^2 + \dots + a_n \cdot 15^n = 5 \\
f(20)-f(15) &= 0 + a_1 \cdot (20-15) + a_2 \cdot (20^2-15^2) + \dots + a_n \cdot (20^n-15^n) = 7-5\\
            &= 5(a_1 + 35a_2 + \dots + (4\cdot 20^{n-1}-3\cdot 15^{n-1}) \cdot a_n) = 2
\end{align*}

In $\mathbb{Z}$, there is no solution to $5x=2$, therefore such a polynomial cannot exist.

\end{document}
