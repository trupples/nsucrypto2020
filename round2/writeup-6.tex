\documentclass[11pt]{llncs}
\usepackage[a4paper]{geometry}
\usepackage{listings}
\usepackage{mathtools}
\usepackage{amsfonts}
\usepackage{amsmath}
\usepackage{hyperref}
\usepackage[dvipsnames]{xcolor}
\hypersetup{
    colorlinks=true,
    linkcolor=RoyalPurple,
    citecolor=black,   
    urlcolor=RoyalPurple
}

\title{NSUCRYPTO 2020: Miller--Rabin revisited}
\author{
	Ioan Dragomir\inst{1} \and
	Gabriel Tulba-Lecu\inst{2} \and
	Mircea-Costin Preoteasa\inst{3}
}

\institute{
	\email{ioandr@gomir.pw} \textendash \ Technical University of Cluj-Napoca \and
	\email{gabi\_tulba\_lecu@yahoo.com} \textendash \ Polytechnic Univeristy of Bucharest \and
	\email{mircea\_costin84@yahoo.com} \textendash \ Polytechnic Univeristy of Bucharest
}


\newtheorem{observation}[theorem]{Observation}

\begin{document}
\let\oldaddcontentsline\addcontentsline
\def\addcontentsline#1#2#3{}
\maketitle
\def\addcontentsline#1#2#3{\oldaddcontentsline{#1}{#2}{#3}}


\let\oldnewpage\newpage
\def\newpage{\hfill}
\setcounter{tocdepth}{3}
\tableofcontents
\def\newpage{\oldnewpage}

\section{Problem summary}

Bob tries to improve a well known probabilistic primality test -- Miller-Rabin. The following modified version of the test is proposed: \\ \\
Let $n$ be the odd number to be tested, we will write it in the form $n - 1 = 2^k3^lm$, where $m$ is not divisible by 2 or 3. 

\[ \begin{aligned}
& \textbf{1.}\ \text{Select a random integer } x \in \{2,\dots,n-2\} \\
& \textbf{2.}\ a \leftarrow x^m \text{ mod } n \\
& \textbf{3.}\ \text{If } a \equiv 1 \text{ (mod n)} \text{ then return "\textbf{probably prime}"}\\
& \textbf{4.}\ \text{For } i = 0,1,\dots,l-1 \text{ do:}\\
& \quad \quad \text{If } a+a^2 \equiv - 1 \text{ (mod n) then return "\textbf{probably prime}"} \\
& \quad \quad a \leftarrow a^3 \text{ (mod n)} \\
& \textbf{5.}\ \text{For } i = 0,1,\dots,k-1 \text{ do:}\\
& \quad \quad \text{If } a \equiv - 1 \text{ (mod n) then return "\textbf{probably prime}"} \\
& \quad \quad a \leftarrow a^2 \text{ (mod n)} \\
& \textbf{6.}\ \text{ return "\textbf{composite}"}
\end{aligned} \]

\newpage
\paragraph{\textbf{Question 1.}} Prove the correctness of the test above (i.e. for any prime number $n$ it will never return "\textbf{composite}").
\paragraph{\textbf{Question 2.}} Does the error probability of falsely classifying a "\textbf{composite}" number $n$ as "\textbf{probably prime}" improve when we switch from the original test, (\textbf{OT}) which has an upper bound on the error probability of $\frac{1}{4}$ \cite{rabin}, to the modified test (\textbf{MT})?

\section{Problem solution}

\subsection{Presenting the original algorithm}
We will keep referencing the \textbf{OT} in this section, so having its pseudocode written here will come in handy: \\ \\
Let $n$ be the odd number to be tested, we will write it in the form $n - 1 = 2^kd$, where $d$ is odd. 

\[ \begin{aligned}
& \textbf{1.}\ \text{Select a random integer } x \in \{2,\dots,n-2\} \\
& \textbf{2.}\ a \leftarrow x^d \text{ mod } n \\
& \textbf{3.}\ \text{If } a \equiv 1 \text{ (mod n)} \text{ then return "\textbf{probably prime}"}\\
& \textbf{4.}\ \text{For } i = 0,1,\dots,k-1 \text{ do:}\\
& \quad \quad \text{If } a \equiv - 1 \text{ (mod n) then return "\textbf{probably prime}"} \\
& \quad \quad a \leftarrow a^2 \text{ (mod n)} \\
& \textbf{5.}\ \text{ return "\textbf{composite}"}
\end{aligned} \]

\subsection{Solving Question 1}
We will execute the algorithm in reverse and prove that if $n$ is prime, then the program will return "\textbf{probably prime}".

The proof is based on Fermat's little theorem: Let $n$ be a prime number, and $x$ s.t. $\gcd(n, x) = 1$, then $x^{n-1} \equiv 1$ (mod n).

Since $n-1 = 2^k3^lm$, then $\dfrac{n-1}{2^i3^j} \in \mathbb{Z}\quad \forall\ 1 \leq i \leq k,\ 0 \leq j \leq l$. From these statements, we get the following result:
\[ \begin{aligned}
& x^{n-1} \equiv 1 \text{ (mod n)} \implies \\
& x^{n-1} -1 \equiv 0 \text{ (mod n)} \implies \\
& (x^{\frac{n-1}{2}} -1)(x^{\frac{n-1}{2}} +1) \equiv 0 \text{ (mod n)} \implies \\
& x^{\frac{n-1}{2}} \equiv \pm 1 \text{ (mod n)}
\end{aligned} \]

If $x^\frac{n-1}{2} \equiv -1$ (mod n), then the \textbf{MT} returns at the last step of loop \textbf{5} and we're done. Otherwise $x^\frac{n-1}{2} \equiv 1$ (mod n) and we must go one step backwards in the algorithm. If $k \geq 2$ we remain in the body of loop \textbf{5}, otherwise we move to loop \textbf{4}. If we remain in loop \textbf{5}, we can repeat this technique: $x^\frac{n-1}{2} \equiv 1 \text{ (mod n)} \implies x^\frac{n-1}{2^2} \equiv \pm 1 \text{ (mod n)}$. This process repeats until we either find $1 \leq i \leq l$ s.t. $x^\frac{n}{2^i} \equiv -1$ (mod n) or we exit loop \textbf{5}, and move backwards to loop \textbf{4}.

The same idea can be applied to loop \textbf{4}. If $l \geq 1 \text{ and } x^\frac{n-1}{2^k} \equiv 1$ (mod n). For simplicity we'll note $n' = \frac{n-1}{2^k}$:

\[ \begin{aligned}
& x^{n'} \equiv 1 \text{ (mod n)} \implies \\
& x^{n'} -1 \equiv 0 \text{ (mod n)} \implies \\
& (x^\frac{n'}{3} -1)((x^\frac{n'}{3})^2 + x^\frac{n'}{3} + 1) \equiv 0 \text{ (mod n)} \implies \\
& x^\frac{n'}{3} \equiv 1 \text{ (mod n) or } (x^\frac{n'}{3})^2 + x^\frac{n'}{3} = -1 \text{ (mod n)}
\end{aligned} \]

Like before, if $(x^\frac{n'}{3})^2 + x^\frac{n'}{3} = -1 \text{ (mod n)}$, then the \textbf{MT} returns at the last step of loop \textbf{4} and we have again finished. Otherwise $x^\frac{n'}{3} \equiv 1$ (mod n) and we must go another step backwards. This will repeat until we either find $1 \leq j \leq l$ s.t. $x^\frac{n'}{3^j} \equiv -1$ (mod n) or we reach the first step of the algorithm. At the first step we have: \\
\[x^\frac{n'}{3^l} \equiv 1\text{ (mod n) }\iff x^\frac{n-1}{2^k3^l} \equiv 1\text{ (mod n) }\iff x^m \equiv 1\text{ (mod n)}\]

Therefore, if we reach the second step of the algorithm going backwards, it will always return "\textbf{probably prime}". Thus, if $n$ is prime the \textbf{MT} must return "\textbf{probably prime}" at some point in its execution, so we have proved the correctness of the algorithm.

\subsection{Solving Question 2}
In \cite{rabin}, M. Rabin proves that the upper bound for the error probability of falsely classifying a "\textbf{composite}" number $n$ as "\textbf{probably prime}" is $\frac{1}{4}$.

This upper bound is achieved for a special class of Carmichael numbers $n = p\cdot q\cdot r$, where $p, q, r$ are prime numbers and $p \equiv q \equiv r \equiv -1$ (mod 4). If there is a number $n$ s.t. $n-1 \not\equiv 0$ (mod 3), then $n - 1 = 2^km$, and so loop \textbf{4} will have no iterations, meaning \textbf{MT} is reduced to \textbf{OT}, so the upper bound for the error probability of \textbf{MT} equals the upper bound for the error probability of \textbf{OT}, which is $\frac{1}{4}$.

In order to find such a number, we wrote a program which found a bunch of them: $10546629279551, 19177682527151, 22799069430611, 52305745012067 $ etc. 

Therefore, we can say that the result M. Rabin obtained is not improved by this modification.

\newpage
\subsection{Bonus result}
As a bonus answer for \textbf{Question 2}, we also found that other Carmichael numbers of the type above have different lower bounds if they are divisible by 3. The experimental result we obtained is that if $n$ is a Carmichael number of the type above and $n-1 = 2^k3^lm$, for m not divisible by 2 and 3, then the upper bound for the error probability of \textbf{MT} is:

\[\frac{1}{4}\cdot (1 - \frac{1}{3} - \\
\frac{1}{3^2} - \dots - \frac{1}{3^l}) = \frac{1}{4}\cdot (1 - \frac{3^l - 1}{2\cdot 3^l}) = \frac{3^l+1}{8\cdot3^l}\]

\appendix
\section{Code for searching special Carmichael numbers}
\scriptsize
\begin{lstlisting}[basicstyle=\ttfamily,language=Python,showstringspaces=false]
from fractions import gcd
# http://oeis.org/A002997 for p, q, r = -1 (mod 4)
w = [8911, 1024651, 1152271, 5481451, 10267951, ...]


def is_witness_orig(p2, d, a, n):
    x = pow(a, d, n)
    if x == 1: return False
    for i in range(p2):
        if (x+1)%n == 0: return False
        x = x * x % n
    return True

def is_witness_mod(p2, p3, d, a, n):
    x = pow(a, d, n)
    if x == 1: return False
    for i in range(p3):
        y = pow(x, 2, n)
        if (x + y + 1) % n == 0:
            return False
        x = x * y % n
    for i in range(p2):
        if x == n-1: return False
        x = x * x % n
    return True

def get_p2_p3_m(n):
    p2 = 0
    p3 = 0
    x = n-1
    while x % 2 == 0:
        x //=2
        p2+=1
    while x%3 == 0:
        x //=3
        p3+=1
    
    return (p2, p3, x)

def count_bad_witnesses(n):
    p2, p3, d = get_p2_p3_m(n)
    cnt_mod = 0
    cnt_org = 0
    total = 0
    for a in range(2, n):
        if(a % 5000000 == 0): # early stop to get an approximation
            return cnt_mod, cnt_org, total
        if(gcd(a, n) != 1):
            continue
        total+=1
        if not is_witness_mod(p2, p3, d, a, n):
            cnt_mod += 1
        if not is_witness_orig(p2, d * 3**p3, a, n):
            cnt_org += 1

    return (cnt_mod, cnt_org, total)

for i in w:
    ce, co, t = count_bad_witnesses(i)
    if ce == i-2 and co == i-2: continue
    print("i:", i, "total:", t, "mod:", ce, "orig:", co, 100 * ce/t, 100 * co/t)
\end{lstlisting}
\normalsize
\medskip
\begin{thebibliography}{9}
\bibitem{rabin} 
Michael O. Rabin
\textit{Probabilistic Algorithm for Testing Primality}.
Institute of Mathematics, Hebrew University, Jerusalem, Israel, and Massachusetts Institute of Technology, Cambridge, Massachusetts 02139.
\end{thebibliography}

\end{document}