\documentclass[11pt]{llncs}
\usepackage[a4paper]{geometry}
\usepackage{mathtools}
\usepackage{amsfonts}
\usepackage{amsmath}
\usepackage{graphics}
\usepackage{colortbl}
\usepackage{listings}
\usepackage{hyperref}
\usepackage{color}
\usepackage[dvipsnames]{xcolor}

\definecolor{albastru}{rgb}{0.81, 0.88, .95}
\definecolor{portocaliu}{rgb}{0.98, 0.89, 0.80}
\hypersetup{
    colorlinks=true,
    linkcolor=RoyalPurple,   
    urlcolor=RoyalPurple
}

\title{NSUCRYPTO 2020: AES-GCM}

\author{
	Ioan Dragomir\inst{1} \and
	Gabriel Tulba-Lecu\inst{2} \and
	Mircea-Costin Preoteasa\inst{3}
}

\institute{
	\email{ioandr@gomir.pw} \textendash \ Technical University of Cluj-Napoca \and
	\email{gabi\_tulba\_lecu@yahoo.com} \textendash \ Polytechnic Univeristy of Bucharest \and
	\email{mircea\_costin84@yahoo.com} \textendash \ Polytechnic Univeristy of Bucharest
}

\begin{document}
\let\oldaddcontentsline\addcontentsline
\def\addcontentsline#1#2#3{}
\maketitle
\def\addcontentsline#1#2#3{\oldaddcontentsline{#1}{#2}{#3}}


\let\oldnewpage\newpage
\def\newpage{\hfill}
\setcounter{tocdepth}{2}
\tableofcontents
\def\newpage{\oldnewpage}

\section{Problem summary}

This problem revolves around the AES-GCM-256 encryption scheme, asking the solver to perform several exploits that arise from the misuse of it, we shall point out that all the attacks are due to nonce reuse:

\paragraph{\textbf{Question 1:}} Given several encrypted messages and the plain-text for one of them, decrypt at least one other message.

\paragraph{\textbf{Question 2}} Given two messages that use the same key and the same nonce, create a message forgery.

\paragraph{\textbf{Question 3}} Keep the construction from \textbf{Q2}, with the exception that the Additional Authenticated Data (\textbf{AAD}) has now 8 unknown bytes $X = f(K)$, for some deterministic function $f$ and the key $K$.

\paragraph{\textbf{Question 4}} Given two messages that use the same key and the same nonce, create a message forgery, such that, the decrypted plain text shall be human readable.

\newpage

\section{Solution}

\subsection{The structure of AES-GCM}
We will start by analyzing a propriety of the GCM mode of operation:

\[ \begin{aligned}
C_i = P_i \oplus E_K(cnt_i) \quad \forall i \in [1,\dots,n]
\end{aligned} \]

Where, $cnt_i$ is the $i^{th}$ value derived from the $IV$ and $n$ is the number of blocks in the plain text.

From this, we can see that GCM acts very similar to a stream cipher. It derives from an $IV$, a stream, and proceeds $XOR$-ing the plain text, with the generated stream. Therefore, if two messages are encrypted under the same $IV$ and $K$, they will be $XOR$-ed with the same stream, so a two time pad attack can be applied.

\subsection{Solution for Question 1}
The first task provided the following information:
\[ \begin{aligned}
& \textbf{1.}\ \text{8 encrypted messages.} \\
& \textbf{2.}\ \text{The plain text for the first message} = \text{"Hello, Bob! How's everything?"} \\
\end{aligned}\]

And asked for the decryption of another plain text. We will begin by performing some reconnaissance on the encrypted messages given. Using a parser written in Python [\nameref{A}], we will separate the various parts of them (encoded in hex):

\begin{table}[]
\resizebox{\columnwidth}{!}{%
\begin{tabular}{|c|c|c|c|c|}
\hline
\textbf{\begin{tabular}[c]{@{}c@{}}Message\\ Number\end{tabular}} & \textbf{Header}  & \textbf{IV}              & \textbf{Authentication Tag}    & \textbf{Encrypted Payload}                                                         \\ \hline
\textbf{0}                                                        & 923488fa5bda1b88 & \cellcolor{albastru} 963ea616e2aeaf3e9b675ecf & b5607a45f8255c9874cfc87bb70c32f8 & 67aada75fa114093bb135d0e21cc73f45f65e8390aca31da97d68ab436                         \\ \hline
\textbf{1}                                                        & fe2a5e45f5d25913 & eb9255003717c7dcea8dd805 & 9e50e2a4f4b2974271301be0e62bede9 & ab4f3df8273b70e3c97153417224e84709af                                               \\ \hline
\textbf{2}                                                        & 9d501a5290c259f7 & c5e80ee2151bafafda16475a & 4a674244d125713ab6fb12ab10649103 & a64276c20d541eb67628aac285d4e791e6fe950c3c3886e246                                 \\ \hline
\textbf{3}                                                        & 89323cbdda2edd8d & 14616e6cedda18beefccdd82 & 036e0fc5e8b4ed98e6fb5ab7ea751226 & 9e8e99fb580e6feb98fc36d4ea49285466                                                 \\ \hline
\textbf{4}                                                        & b09c69d1ea5d3d10 & fa1adf74f5f48c01887821e3 & 48ae7dfd21f43a6e2ee1adb00f38e20c & 12a4ec16c1262c680e1c0e506f85ce3083e24dad5345f0e4a2ffa5                             \\ \hline
\textbf{5}                                                        & 42e76d8f00a5ed7a & \cellcolor{albastru} 963ea616e2aeaf3e9b675ecf & 29b5ebab8685b8bb4a0bec18d8f5d50c & \cellcolor {portocaliu} 63a6d87afa510ef184100e45458335e31674b861                                           \\ \hline
\textbf{6}                                                        & f54c87dd01d3d9a9 & \cellcolor{albastru} 963ea616e2aeaf3e9b675ecf & 6609632220fd69c2fe4c5fc1f5f1f13b & 61a0c56df45107b8b551155d49c224b65820ff210ed468c390cb8db529e69c44ed3e7dde41dc20fd85 \\ \hline
\textbf{7}                                                        & 26cfefbc3df253ea & c5e80ee2151bafafda16475a & f115d23cd40ef8adee22f862a79c5d04 & a10d3bda0b424daf6667add6cbdae1d0e9e6d95b                                           \\ \hline
\end{tabular}
}
\end{table}

From the table we observe that messages 5 and 6 share their IV with message 0, for which we know the plain text. Furthermore, message 5 has a smaller length than that of message 0. From this,
we can apply the idea presented in the previous section:
\[\begin{aligned}
& C_0 = P_0 \oplus \text{stream}(IV, K) \implies\\
& \text{stream}(IV, K) = P_0 \oplus C_0 \\
& C_5 = P_5 \oplus \text{stream}(IV, K) \implies \\
& P_5 = C_5 \oplus P_0 \oplus C_0\\
\end{aligned}
\]

Since we know $C_5, P_0, C_0$, we can decrypt message 5. This message turns out to be: \textbf{Lincoln Park, 10:15.}. As a side note, we can only partially decrypt message 6, because we don't know all of the stream that it was $XOR$-ed with. The partially decrypted message 6 is: \textbf{Nostalgia is a eternal motif}.

\subsection{Preforming message forgery}
In this section we will present an attack that will solve the remaining $3$ questions. This attack fully breaks AES-GCM's integrity, when a nonce is reused, being able to compute authenticated tags for an arbitrary cipher text. Keep in mind that all the following operations are over the field $GF(2^{128})$.

Let's start by making some notations:
\[\begin{aligned}
& A = Header | IV = A_1 | A_2 \\
& C = C_1 | C_2 | \dots | C_n \\
& L = len(A) | len(C) \\
& S = E_K(cnt_0) \\
& H = E_K(0)
\end{aligned}
\]

Now we can express the authentication tag as the following sum:
\[tag = A_1 \cdot H^{n+3} + A_2 \cdot H^{n+2} + C_1 \cdot H^{n+1} + \dots + C_n \cdot H^2 + L \cdot H + S\]

Does this look familiar? Well, it should, it is a polynomial, evaluated at point $H$.

Let's continue by looking at what we don't know. We do not know $S$ and $H$. Since $S$ is the constant term of the polynomial and it depends only on $IV$ and $K$, we can get read of it if we have a nonce reuse. Let $P_1$ and $P_2$ be polynomials for the tag of two distinct messages, that use the same $IV$ and $K$:
\[ \begin{aligned}
& P_1(X) = A_1 \cdot X^{n+3} + A_2 \cdot X^{n+2} + C_1 \cdot X^{n+1} + \dots + C_n \cdot X^2 + L \cdot X + S\\
& P_2(X) = A_1' \cdot X^{n+3} + A_2' \cdot X^{n+2} + C_1' \cdot X^{n+1} + \dots + C_n' \cdot X^2 + L' \cdot X + S\\
& P(X) = P_1(X) + P_2(X) + tag_1 + tag_2 = (A_1 + A_1') \cdot X^{n+3} + (A_2 + A_2') \cdot X^{n+2} + \\ & (C_1 + C_1') \cdot X^{n+1} + \dots + (C_n + C_n') \cdot X^2 + (L + L') \cdot X + tag_1 + tag_2\\
\end{aligned}
\]

Since $P(H) = 0$, then $H$ is a root of the polynomial so we can recover it by finding the roots of $P(X)$. After finding $H$, we can find $S$ from the following relation:
\[S = P1(H) + tag_1\]

Now that we know how to find both $H$ and $S$, that means we can compute tags for arbitrary cipher texts. We can now move to applying this attack and solving the remaining questions.

\newpage
\subsection{Solution for Questions 2 and 4}

In order to find a solution for Question 2, we simply have to apply the algorithm described in the previous section. As we have mentioned above, being able to forge authentication tags for an arbitrary cipher text, allows us to modify any bit from a message. Furthermore, since we want to also answer the Question 4, we will make the following observation.
\[ \text{Human readable text has ASCII codes } \in [32,\dots,127]\]

This means that, for a human readable byte if we only flip the least significant $5$ bits, the newly obtained byte will still be human readable.

Thus we have an answer for Question 4: If we only modify the least significant $5$ bits of a byte in the cipher text, then the plain text will still be human readable. In our implementation of the attack, as a proof of concept we only changed the least significant bit of every $4^{th}$ byte. The new cipher text(including the header, IV and tag) we obtained is:
\[\begin{aligned}
& \texttt{fff6ab48fa2d6c404aec77d7184536520e4e2}\\                                                                                   & \texttt{04b2f84d173ba889601957cb45420f7bffb99}\\                                                                                   & \texttt{50620360da94bfe55da98194e06a233ad34db}\\                                                                                   & \texttt{aa9c54b8fdd03f96c0a808e32aedafcd8b213}\\                                                                                   & \texttt{5389a080b9831058738abc9cea54fcb3945cb}\\                                                                                   & \texttt{9abb4973be2f4e769ca5e78899b4a8b5c0195}\\                                                                                   & \texttt{df1aba3b55b8a12acdb434e3b95225f4d94d0}\\                                                                                   & \texttt{80509ee8a543670a9a5f80a0c716c0dafcef9}\\                                                                                   & \texttt{d72dcb9ec42015c530bbbe11b4a8e2cb8a42f}\\                                                                                   & \texttt{60f1aead93aa2ddd113f4e106243dc2c4b421}\\                                                                                   & \texttt{ec5f01aa805f8b0ae5eca6f5f30939482f04f}\\                                                                                   & \texttt{986c812588b7cd93b50c21ade66512cbe86af}\\                                                                                   & \texttt{63d6aed783cc750416d06eda8c8426269eaf3}\\                                                                                   & \texttt{ab307b2fa100cac880ffe6b73131d91acea36}\\                                                                                   & \texttt{6a65d1905cb3680b1ee69b5cc215522775672}\\                                                                                   & \texttt{6ce5242f851d4aa2ddc19a9fff9594edb1791}
\end{aligned}\]

\newpage
\subsection{Solution for Question 3}

The difference between Question 3 and Question 2 is the Additional Authenticated Data. Question 3's AAD has $8$ unknown bytes derived from the key, appended to the AAD of Question 2. This could be a problem but it fortunately isn't. The organisers have chosen the two messages that have the same nonce such that they have the same length too! 

This means that the tag polynomials $P_1(X)$ and $P_2(X)$ have the same degree and the newly added value $X = f(K)$ in the AAD is on the same position so it will cancel itself in the context of a nonce reuse, also this value is dependent only on the key $K$. Thus, we can think of this new added value as another part of the constant term $S$, and hence the attack from Question 2 can be applied here as well. So after adding 8 "0" bytes (any value would work here since the values will cancel each other) to the original AAD, we can apply the exact same algorithm as in Question 2.
The new cipher text(including the header, IV and tag) we obtained is:
\[\begin{aligned}
& \texttt{047b8c4ede1735befe4deb8b8e8212a00bddd1ec21c731}\\                                                             & \texttt{2b93dc64db60ce9454827c3bb7fac7b490d89bec777dff}\\                                                             & \texttt{ca91744251e2e6288c586172ad5c3ea9481d17644a5e6a}\\                                                             & \texttt{9f86219482bc1f898f1c592e40d16d9446b9f343319d78}\\                                                             & \texttt{5d7e2ac34be60fbbddc9943df5bc1cd5867ff7557982f2}\\                                                             & \texttt{d1960a8d5ed293bffb48cc8b99609684dc211d3d537b1a}\\                                                             & \texttt{51b6173c7cc036096663e193acacc56b3a8cb0a066544a}\\                                                             & \texttt{b69820a1799545459a369b71aa9870b4595ccc6856e37a}\\                                                             & \texttt{4aa6c94fc6a1309d0a8c547ce9b2ccd8769037f5adc2c0}\\                                                             & \texttt{65a9405f396c00b6adea76faa8280c064bf9ad1eb0c883}\\                                                             & \texttt{27522448a6f002541566f30e690bfcd5d90a0ac81f64b8}\\                                                             & \texttt{9bafda35aa70f985051fd2d1917377d5e64021db9ad463}\\                                                             & \texttt{e6d6c910e4aedacf54d768a14de2ebd2296c08697d04f3}\\                                                             & \texttt{f52d600a72ea96bb25d4201fac9b058b6bbd086253dc52}\\                                                             & \texttt{81eaba723d49a0bafb58f4eaf31c143502d2b80532bfc1}\\                                                             & \texttt{1e736db95c5f78a08e21e0}\\
\end{aligned}\]

\newpage
\subsection{Implementation}
We have implemented this attack in C++, using the libraries \href{https://gmplib.org/}{GMP} and \href{https://www.shoup.net/ntl/}{NTL}. The source code for the attack (\textbf{utils.cpp, utils.h \& main.cpp}) has been attached to this solution. Also some scripts written in Python3 for testing the correctness of our solution have been attached.
\appendix
\section{A}
\label{A}
\begin{lstlisting}
import binascii


def xor(a, b):
    o = b''
    for x, y in zip(a, b):
        o += bytes([(x ^ y)])
    return o


def split_message(msg):
    header = msg[:8]
    IV = msg[8:20]
    tag = msg[-16:]
    enc = msg[20:-16]
    return (header, IV, enc, tag)


def read_messages():
    msgs = []
    for i in range(8):
        msg = open("{}.message".format(i), 'rb').read()
        msgs.append(split_message(msg))
    return msgs


def print_message(msg):
    header, IV, enc, tag = msg
    print('header: {} IV: {} tag: {} enc: {}'.format(binascii.hexlify(header), binascii.hexlify(IV),
                                                     binascii.hexlify(tag), binascii.hexlify(enc)))


msgs = read_messages()
for msg in msgs:
    print_message(msg)

known_IV = binascii.unhexlify('963ea616e2aeaf3e9b675ecf')
msg_0 = b"Hello, Bob! How's everything?"
enc_0 = binascii.unhexlify('67aada75fa114093bb135d0e21cc73f45f65e8390aca31da97d68ab436')
known_stream = xor(msg_0, enc_0)

for msg in msgs:
    header, IV, enc, tag = msg
    if IV == known_IV:
        print(xor(enc, known_stream))
\end{lstlisting}

\end{document}
