\documentclass[11pt]{llncs}
\usepackage[a4paper]{geometry}
\usepackage{mathtools}
\usepackage{amsfonts}
\usepackage{amsmath}
\usepackage{hyperref}
\usepackage[dvipsnames]{xcolor}
\hypersetup{
    colorlinks=true,
    linkcolor=RoyalPurple,   
    urlcolor=RoyalPurple
}

\title{NSUCRYPTO 2020: Orthomorphisms}
\author{
	Ioan Dragomir\inst{1} \and
	Gabriel Tulba-Lecu\inst{2} \and
	Mircea-Costin Preoteasa\inst{3}
}

\institute{
	\email{ioandr@gomir.pw} \textendash \ Technical University of Cluj-Napoca \and
	\email{gabi\_tulba\_lecu@yahoo.com} \textendash \ Polytechnic Univeristy of Bucharest \and
	\email{mircea\_costin84@yahoo.com} \textendash \ Polytechnic Univeristy of Bucharest
}


\newcommand{\dm}[0]{$\text{D}_{2^m}$\hspace{0.7mm}}
\newcommand{\dM}[1]{$\text{D}_{2^{#1}}$\hspace{0.7mm}}

\newtheorem{observation}[theorem]{Observation}

\begin{document}
\let\oldaddcontentsline\addcontentsline
\def\addcontentsline#1#2#3{}
\maketitle
\def\addcontentsline#1#2#3{\oldaddcontentsline{#1}{#2}{#3}}


\let\oldnewpage\newpage
\def\newpage{\hfill}
\setcounter{tocdepth}{3}
\tableofcontents
\def\newpage{\oldnewpage}

\section{Problem summary}

The dihedral group of order $2^m$, denoted \dm, $m \geq 4$, is generated by $a$, $u$ so that:
\[ a^{2^{m-1}} = e, \quad u^2 = e, \quad ua = a^{-1}u \]

Intuitively, it describes the symmetries of a regular polygon with $2^{m-1}$ vertices and two distinct faces, where $a$ is a rotation which circularly permutes the vertices by one position, and $u$ flips it over. By applying $a$ $2^{m-1}$ times, we do a full rotation and end up at the identity element. Flipping twice has no effect. Applying a rotation ($a^k$) and then flipping ($u$) is equivalent to flipping ($u$) and them applying the opposite rotation ($a^{-k}$).

Bob proposes the following morphism family, parameterised by:
\[ r_1, r_2, c_1, c_2, q_1, q_2, b_1, b_2 \in \{0, \dots, 2^{m-1}-1\} \]
\[ \theta^{(r_1,r_2,c_1,c_2)}_{(q_1,q_2,b_1,b_2)}(a^i) = \begin{cases}\begin{aligned}
	a^{r_1i+c_1},\quad &\text{if } 0 \leq i < 2^{m-2} \\
	a^{r_2i+c_2}u,\quad &\text{if } 2^{m-2} \leq i < 2^{m-1}
\end{aligned}\end{cases} \]
\[ \theta^{(r_1,r_2,c_1,c_2)}_{(q_1,q_2,b_1,b_2)}(a^iu) = \begin{cases}\begin{aligned}
	a^{q_2i+b_1}u,\quad &\text{if } 0 \leq i < 2^{m-2} \\
	a^{q_2i+b_2},\quad &\text{if } 2^{m-2} \leq i < 2^{m-1}
\end{aligned}\end{cases} \]
$\theta$ is an orthomorphism if the mapping $\pi: x \rightarrow x^{-1}\theta(x)$ is a permutation on the respective group. In other words, there are no two distinct $x_1, x_2$ for which $x_1^{-1}\theta(x_1) = x_2^{-1}\theta(x_2)$. 

While the visual complexity of the last few ideas may seem daunting, after some closer inspection, they become much tamer beasts.

\paragraph{\textbf{Question 1:}} Describe all orthomorphisms $\theta$ in \dM{4} and find their number.

\paragraph{\textbf{Question 2:}} For each $m \geq 4$, describe all orthomorphisms $\theta$ in \dm.

\section{Solution}

\subsection{Exploring the \dm structure}

We start by studying the product of two \dm elements, 

\[ \begin{aligned}
	a^i&  & \cdot\ a^j&  & =\ &a^{i+j} \\
	a^i&u & \cdot\ a^j&  & =\ &a^{i-j}u \\
	a^i&  & \cdot\ a^j&u & =\ &a^{i+j}u \\
	a^i&u & \cdot\ a^j&u & =\ &a^{i-j}
\end{aligned} \]
the inverse of an element,
\[ \begin{aligned}
(a^i)^{-1} &= a^{-i} = a^{2^{m-1}-i} \\
(a^iu)^{-1} &= a^iu
\end{aligned} \]
and a reduced formula for $\pi(x) \stackrel{def}{=} x^{-1}\theta(x)$:
\[ \begin{aligned}
&\pi(a^i)\hphantom{u} = (a^i)^{-1} \theta(a^i) = \begin{cases}
	a^{-i} a^{r_1i+c_1}\\
	a^{-i} a^{r_2i+c_2}u
\end{cases} = \begin{cases}
	a^{r_1i+c_1-i}, & \text{if } i < 2^{m-2} \\
	a^{r_2i+c_2-i}u, & \text{if } i \geq 2^{m-2}
\end{cases} \\
&\pi(a^iu) = (a^iu)^{-1} \theta(a^iu)
 = \begin{cases}
	a^iu a^{q_1i+b_1}\\
	a^iu a^{q_2i+b_2}u
\end{cases} = \begin{cases}
	a^i a^{-q_1i-b_1} u, & \text{if } i < 2^{m-2} \\
	a^i a^{-q_2i-b_2} uu, & \text{if }  i \geq 2^{m-2}
\end{cases} \\
&\pi(a^i)\hphantom{u} = \begin{cases}
	a^{i(r_1-1)+c_1}, & \text{if } i < 2^{m-2} \\
	a^{i(r_2-1)+c_2}u, & \text{if } i \geq 2^{m-2}
\end{cases} \\
&\pi(a^iu) = \begin{cases}
	a^{i(1-q_1)-b_1}u, & \text{if } i < 2^{m-2} \\
	a^{i(1-q_2)-b_2}, & \text{if }  i \geq 2^{m-2}
\end{cases}
\end{aligned} \]

\newpage
\subsection{Splitting the problem in half}

We can separate all \dm elements into two classes: \[ \begin{aligned}
& A = \{a^i\ |\ 0 \leq i < 2^{m-2}\} \cup \{a^iu\ |\ 2^{m-2} \leq i < 2^{m-1}\} \\
& B = \{a^iu\ |\ 0 \leq i < 2^{m-2}\} \cup \{a^i\ |\ 2^{m-2} \leq i < 2^{m-1}\}
\end{aligned} \]

\begin{observation} \label{split-inject} \hfill\\
When put through $\pi$, any element of $A$ will become something of the form $a^i$, and any element of $B$ will become $a^iu$. \end{observation}

\begin{observation} \label{split-indep} \hfill\\
If $(r_1, c_1, q_2, b_2)$ produce a bijective mapping from $A$ to $\{a^i\ |\ 0 \leq i < 2^{m-1}\}$, and so do $(r_2, c_2, q_1, b_1)$ from $B$ to $\{a^iu\ |\ 0 \leq i < 2^{m-1}\}$, the morphism $\theta$, given by their combination, will be bijective from \dm to \dm. Thus the two sub-problems are independent. \end{observation}

We will study the sub-problem $(r_1,c_1,q_2,b_2)$ and change our focus from $A$ to $\mathbb{Z}_{2^{m-1}}$, by considering the following mapping, which "extracts" $a$'s exponent, as well as its inverse which constructs an element of $A$ given its $a$ exponent:

\[ \begin{aligned}
& f_A: A \to \mathbb{Z}_{2^{m-1}} \\
& f_A(a^iu^j) = i,\quad i \in \{0, \dots, 2^{m-1}\},\quad j \in \{0,1\} \\
& f_A^{-1}(i) = \begin{cases}
	a^i, & \text{if } i < 2^{m-2} \\
	a^iu, & \text{if } i \geq 2^{m-2}
\end{cases}\\
\end{aligned} \]
These help us define a function $g_A$ which describes how $\pi$ acts on the $a$ exponents of elements in $A$: 
\[ \begin{aligned}
& g_A: \mathbb{Z}_{2^{m-1}} \to \mathbb{Z}_{2^{m-1}} \\
& g_A(i) = f_A(\pi(f_A^{-1}(i))) \\
& g_A(i) = \begin{cases}
	f_A(\pi(a^i)) \\
	f_A(\pi(a^iu))
\end{cases} = \begin{cases}
	f_A(a^{i(r_1-1)+c_1}), & \text{if } i < 2^{m-2} \\
	f_A(a^{i(1-q_2)-b_2}), & \text{if }  i \geq 2^{m-2}
\end{cases} \\
& g_A(i) = \begin{cases}
	i(r_1-1)+c_1, & \text{if } i < 2^{m-2} \\
	i(1-q_2)-b_2, & \text{if }  i \geq 2^{m-2}
\end{cases}
\end{aligned} \]

Similarly, if we define these functions' counterparts for $B$:
\[ \begin{aligned}
& f_B: B \to \mathbb{Z}_{2^{m-1}}\\
& f_B(a^iu^j) = i \\
& f_B^{-1}(i) = \begin{cases}
	a^iu, & \text{if } i < 2^{m-2} \\
	a^i, & \text{if } i \geq 2^{m-2}
\end{cases} \end{aligned} \]

\[ \begin{aligned}
& g_B: \mathbb{Z}_{2^{m-1}} \to \mathbb{Z}_{2^{m-1}}\\
& g_B(i) = f_B(\pi(f_B^{-1}(i))) \\
& g_B(i) = \begin{cases}
	i(r_2-1)+c_2, & \text{if } i < 2^{m-2} \\
	i(1-q_1)-b_1, & \text{if }  i \geq 2^{m-2}
\end{cases}
\end{aligned} \]

\begin{observation} \hfill\\
$g_A$ and $g_B$ have identical structures. Given any solution $(r_1, c_1, q_2, b_2)$ for which $g_A$ is bijective, the same parameters will also make $g_B$ bijective, and vice versa. Thus, the two sub-problems are equivalent.
\end{observation}

From now on, we will study a sub-problem $(r,c,q,b)$ which has a solution if $g$ is bijective:
\[ \begin{aligned}
& g: \mathbb{Z}_{2^{m-1}} \to \mathbb{Z}_{2^{m-1}} \\
& g(i) = \begin{cases}
	i(r-1)+c, & \text{if } i < 2^{m-2} \\
	i(1-q)-b, & \text{if }  i \geq 2^{m-2}
\end{cases}
\end{aligned} \]

\subsection{Solving the sub-problem}

We continue by further subdividing the problem into three cases, based on the congruence of $r-1$ and $1-q$ modulo 4:

\subsubsection{Multiples of 4}
\[ \begin{aligned}
& r-1 \equiv 0 \quad (\text{mod }4) \quad\Rightarrow\quad g(0) = g(2^{m-3}) = c \\
& 1-q \equiv 0 \quad (\text{mod }4) \quad\Rightarrow\quad g(2^{m-2}) = g(2^{m-3}+2^{m-2}) = -b
\end{aligned} \]
If either is true, $g$ is not a bijection. Thus no solutions exist for $r = 4k+1$ or $q = 4k+1$.

\subsubsection{Other multiples of 2, i.e. $4k+2$}\hfill\\
	
Studying the first branch of $g$, (i.e. $0 \leq i < 2^{m-2}$) we show that $i(r-1)+c$ covers only elements with the same parity as c:
\[ \begin{aligned}
i(r-1)+c\ \equiv\ i(4k+2)+c\ \equiv\ 2i(2k+1)+c\ \equiv\ c\quad (\text{mod }2)
\end{aligned} \]
and that all of those elements are visited. We can prove this by contradiction. Suppose there exist $i_1 \neq i_2$ such that $g(i_1) = g(i_2)$, then:
\[ \begin{aligned}
& i_1(r-1)+c - i_2(r-1)-c = \alpha 2^{m-1}\text{, for some }\alpha\in \mathbb{Z} \\
& i_1(r-1) - i_2(r-1) = \alpha 2^{m-1} \\
& (r-1)(i_1-i_2) = \alpha 2^{m-1} \\
& (4k+2)(i_1-i_2) = \alpha 2^{m-1}\text{, for some }k\in \mathbb{Z} \\
& (2k+1)(i_1-i_2) = \alpha 2^{m-2} \\
 \iff & (2k+1)(i_1-i_2) \equiv 0\quad (\text{mod }2^{m-2})
\end{aligned} \]
$2k+1$ is invertible modulo $2^{m-2}$, because $\gcd(2k+1, 2^{m-2}) = 1$, therefore the only solution is when $i_1 - i_2 \equiv 0\quad (\text{mod }2^{m-2})$. Since $i_1, i_2 < 2^{m-2}$, $i_1 = i_2$, which is a contradiction.

By the same logic, the branch $i(1-q)-b,\ 2^{m-2} \leq i < 2^{m-1}$ also maps to all elements with the same parity as $b$.

\paragraph{Sufficiency:}
If $r-1$ and $1-q$ are both of the form $4k+2$, then $g$ is a bijection if $b$ and $c$ have opposite parities, so that the image of one branch is all the even elements, and the image of the other branch is all the odd elements.

\paragraph{Necessity:}
If $r-1$ is of the form $4k+2$, then $1-q$ must also be. Otherwise, it's either $4k$, which we proved to be invalid, or $2k+1$. In the latter case, $i(1-q)+b$ will cover elements of both parities. As $i(r-1)+c$ covers all elements of some parity, there will be at least some overlapping between the two branches' images.  Therefore the only option for $1-q$ is $4k+2$. If we fix $1-q$ to have this form, $r-1$ is also forced to be the same.

\subsubsection{Odd numbers}\hfill\\
The only case left is when $r-1$ and $1-q$ are both odd.\\

Given a linear function $F(x)=ax+b$ on $x \in \{0,\ \dots,\ 2^{m-2}-1\}$, we can create $F'(x)=-ax+a2^{m-2}-a+b$ which generates the same results, but in opposite order ($F'(x) = F(2^{m-2}-x-1)$).

\begin{observation}\label{two-linear}
Given an image which is generated by a linear function from $\mathbb{Z}_{2^{m-2}}$ to $\mathbb{Z}_{2^{m-1}}$, there are exactly 2 linear functions which generate that image, of the forms $F(x)$ and $F'(x)$.
\end{observation}

We fix the first branch of $g$ to be $ax+b$. Since $a$ is odd, we know it covers exactly half of $\mathbb{Z}_{2^{m-1}}$. The image of the second branch, let it be $B$, must then correspond to all the additive inverses of the values generated by the first branch.

One linear function on $x \in \{2^{m-2},\ \dots,\ 2^{m-1}-1\}$  which has the image $B$ is ${ax+b}$ once again. Therefore there must be only one other linear function which generates the image, but in reversed order: $-ax-a+b$. It follows from Obs. \ref{two-linear} that there are no other linear functions on $x \in \{2^{m-2},\ \dots,\ 2^{m-1}-1\}$ which generate $B$.

Going back to the initial notation for $g$, we have two valid bijections:
\[ \begin{aligned}
& g_1(i) = \begin{cases}
	(r-1)i + c,\quad\text{if }0 \leq x < 2^{m-2} \\
	(r-1)i + c,\quad\text{if }2^{m-2} \leq x < 2^{m-1} 
\end{cases} \\
& \quad\Rightarrow\quad q_1 \equiv -r,\ b_1 \equiv -c \\\\
&g_2(i) = \begin{cases}
	(r-1)i + c,\quad\text{if }0 \leq x < 2^{m-2} \\
	-(r-1)i + 2^{m-2} -r+1+c,\quad\text{if }2^{m-2} \leq x < 2^{m-1}
\end{cases} \\
& \quad\Rightarrow\quad q_2 \equiv r,\ b_2 \equiv -2^{m-2}+r-c-1
\end{aligned} \]

\subsection{Question 2: Describing and counting all $\theta$ orthomorphisms}

The sub-problem $(r,c,q,b)$ has the following solution cases:\\\\
$(4i+3,\ j,\ 4k+3,\ l),\quad 0 \leq i, k < 2^{m-3},\quad 0 \leq j, l < 2^{m-1},\quad\text{p}(j)\neq\text{p}(l)$\marginpar{$\text{p}(2k)=0$\\$\text{p}(2k+1)=1$}\\
$\hphantom{ }\quad\quad\quad\Rightarrow\quad 2^{m-3} \cdot 2^{m-1} \cdot 2^{m-3} \cdot 2^{m-2} = 2^{4m-9}$ solutions.\\\\
$(2i+1,\ j,\ -2i-1,\ -j),\quad 0 \leq i < 2^{m-2},\quad 0 \leq j < 2^{m-1}$\\
$\hphantom{ }\quad\quad\quad\Rightarrow\quad\ 2^{m-2} \cdot 2^{m-1} = 2^{2m-3}$ solutions.\\\\
$(2i+1,\ j,\ 2i+1,\ 2^{m-2}+2i+1-j),\quad 0 \leq i < 2^{m-2},\quad 0 \leq j < 2^{m-1}$\\
$\hphantom{ }\quad\quad\quad\Rightarrow\quad\ 2^{m-2} \cdot 2^{m-1} = 2^{2m-3}$ solutions.\\\\
For a total of $2^{4m-9} + 2^{2m-3} + 2^{2m-3} = 2^{2m-2}(2^{2m-7}+1)$ solutions for the sub-problem. The orthomorphisms $\theta$ are decomposed into two independent such sub-problems, so the number of $\theta$ solutions is the square of the number of solutions to the sub-problems, or

\[ (2^{2m-2}(2^{2m-7}+1))^2 = 2^{4m-4}(2^{4m-14} + 2^{2m-6} + 1) = \]
\LARGE \[ \mathbf{ 2^{8m-18} + 2^{6m-10} + 2^{4m-4} } \] \normalsize

\subsection{Question 1}

In the special case $m=4$, $\theta^{(r_1, r_2, c_1, c_2)}_{(q_1, q_2, b_1, b_2)}$ is an orthomorphism when both $(r_1, c_1, q_2, b_2)$ and $(r_2, c_2, q_1, b_1)$ are of one of the forms:

\[ \begin{aligned}
& (4i+3,\ j,\ 4k+3,\ l),& & i, k \in \{0,1\},\quad j, l \in \{0,\dots,7\},\quad\text{p}(j)\neq\text{p}(l) \\
& (2i+1,\ j,\ -2i-1,\ -j),& & i \in \{0,1,2,3\},\quad j \in \{0,\dots,7\} \\
& (2i+1,\ j,\ 2i+1,\ 5+2i-j),& & i \in \{0,1,2,3\},\quad j \in \{0,\dots,7\}
\end{aligned} \]

The number of solutions is $2^{8\cdot 4-18}+2^{6\cdot 4-16}+2^{4\cdot 4-4}=36864$

\end{document}
